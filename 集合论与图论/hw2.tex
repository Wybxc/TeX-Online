\documentclass{ctexart}
\usepackage{amsmath, amsfonts, amssymb}
\usepackage{geometry}
\special{dvipdfmx:config z 0} % delete this when release

\geometry{a4paper,scale=0.8}

\title{集合论与图论~作业2}
\author{庄嘉毅}
\date{September 2022}

\def\QED{\hfill $\square$}
\def\st{\textrm{s.t.}\,}
\DeclareMathOperator{\e}{e}

\everymath{\displaystyle}
\linespread{2}

\begin{document}

\maketitle

\paragraph*{1} 令 $F(x):$ x 是乌龟,$G(x):$ x 是兔子,$H(x,y):$ x 比 y 跑得快。

命题符号化为 $\exists x (\ F(x)\rightarrow \exists y(\ G(y)\rightarrow H(x,y)\ )\ )$。

求前束范式:
\begin{align*}
\exists x (\ F(x)\rightarrow \exists y(\ G(y)\rightarrow H(x,y)\ )\ ) &\Leftrightarrow \exists x \exists y (\ F(x)\rightarrow (\ G(y)\rightarrow H(x,y)\ )\ ) \\
&\Leftrightarrow \exists x \exists y (\ F(x)\wedge G(y) \rightarrow H(x,y)\ )
\end{align*}

\paragraph*{2} 取个体域为实数集合,$A(x): x>1$,$B(x):x>2$。

此时左侧为假,右侧为真。

\end{document}
