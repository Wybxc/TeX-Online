\documentclass{ctexart}
\usepackage{diagbox}
\usepackage[table,xcdraw]{xcolor}
\usepackage{amsmath, amsfonts, amssymb}
\usepackage{geometry}
\special{dvipdfmx:config z 0} % delete this when release

\geometry{a4paper,scale=0.8}

\title{集合论与图论~作业1}
\author{庄嘉毅}
\date{September 2022}

\def\QED{\hfill $\square$}
\def\st{\textrm{s.t.}\,}
\DeclareMathOperator{\e}{e}

\everymath{\displaystyle}
\linespread{2}

\begin{document}

\maketitle

以下以行表示 $A$ 的取值,列表示 $B$ 的取值。

\begin{table}[ht]
    \centering
    \begin{tabular}{| >{\columncolor[HTML]{E0E0E0}}c|l|l|}
        \hline
        \rowcolor[HTML]{E0E0E0}
        \diagbox{$A$}{$B$} & 0 & 1 \\ \hline
        0                  & 1 & 1 \\ \hline
        1                  & 1 & 0 \\\hline
    \end{tabular}
    \caption{$\neg(A\wedge B)$的真值表}
\end{table}

\begin{table}[ht]
    \centering
    \begin{tabular}{| >{\columncolor[HTML]{E0E0E0}}c|l|l|}
        \hline
        \rowcolor[HTML]{E0E0E0}
        \diagbox{$A$}{$B$} & 0 & 1 \\ \hline
        0                  & 1 & 1 \\ \hline
        1                  & 1 & 0 \\\hline
    \end{tabular}
    \caption{$\neg A\vee \neg B$的真值表}
\end{table}

因此 $\neg (A \wedge B) \Leftrightarrow \neg A \vee \neg B$。

\begin{table}[ht]
    \centering
    \begin{tabular}{| >{\columncolor[HTML]{E0E0E0}}c|l|l|}
        \hline
        \rowcolor[HTML]{E0E0E0}
        \diagbox{$A$}{$B$} & 0 & 1 \\ \hline
        0                  & 1 & 0 \\ \hline
        1                  & 0 & 0 \\\hline
    \end{tabular}
    \caption{$\neg(A\vee B)$的真值表}
\end{table}

\begin{table}[ht]
    \centering
    \begin{tabular}{| >{\columncolor[HTML]{E0E0E0}}c|l|l|}
        \hline
        \rowcolor[HTML]{E0E0E0}
        \diagbox{$A$}{$B$} & 0 & 1 \\ \hline
        0                  & 1 & 0 \\ \hline
        1                  & 0 & 0 \\\hline
    \end{tabular}
    \caption{$\neg A\wedge \neg B$的真值表}
\end{table}

因此 $\neg (A \vee B) \Leftrightarrow \neg A \wedge \neg B$。

\end{document}
