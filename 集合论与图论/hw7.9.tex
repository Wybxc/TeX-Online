\documentclass{ctexart}
\usepackage{multicol}
\usepackage{amsmath, amsfonts, amssymb}
\usepackage{tikz}
\usepackage{geometry}
\special{dvipdfmx:config z 0} % delete this when release

\usetikzlibrary{positioning}

\geometry{a4paper,scale=0.8}

\title{集合论与图论~作业7-9}
\author{庄嘉毅}
\date{October 2022}

\def\QED{\hfill $\square$}
\def\st{\textrm{s.t.}\,}
\def\pair#1{\left\langle #1 \right\rangle}
\def\conj{\mathrel{\wedge}}
\def\disj{\mathrel{\vee}}
\def\equ{\mathrel{\Leftrightarrow}}
\def\restr{\mathbin{\upharpoonright}}
\def\ple{\mathrel{\preccurlyeq}}
\DeclareMathOperator{\dom}{dom}
\DeclareMathOperator{\ran}{ran}
\DeclareMathOperator{\fld}{fld}
\DeclareMathOperator{\card}{card}
\DeclareMathOperator{\e}{e}

\everymath{\displaystyle}
% \linespread{2}

\begin{document}

\maketitle

\section*{习题三}

\paragraph*{11} 证明: 设存在 $b,b'$ 使得 $g(b)=g(b')$.
由 $f$ 是满射, 知 $g(b)\ne \varnothing$, 因此
$\forall x\in A, f(x)=b \equ f(x)=b'$. 由 $f$ 的单值性,
知 $b=b'$, 因此 $g$ 是单射. \QED

\paragraph*{19}

\subparagraph*{(1)} $f(A_1)=\{1,2,3\}$,
$f_{-1}(B_1)=\{0, 4, 5, 6\}$,

\subparagraph*{(2)} $g(A_2)=\mathbb{N}$,
$g_{-1}(B_2)=\{3,6\}$.

\subparagraph*{(3)} $f$ 是单射, 有反函数. $g$ 不是单射, 无反函数.

\paragraph*{20}

\subparagraph*{(1)} 设存在 $b, b'\in B$ 使得 $f(b)=f(b')$.
由 $g$ 是满射, 故存在 $a,a'\in A$ 使得 $g(a)=b, g(a')=b'$,
那么 $f(g(a))=f(g(a'))$.
又 $f\circ g$ 是单射, 因此 $a=a'$, 故 $b=b'$, 所以 $f$ 是单射. \QED

\subparagraph*{(2)} 任取 $b\in B$, 那么 $c=f(b)\in C$.
由 $f\circ g$ 是满射, 存在 $a\in A$, 使得 $f(g(a))=c$.
又由 $f$ 是单射, 知 $g(a)=b$, 因此 $g$ 是满射. \QED

\section*{习题四}

\paragraph*{2}

\subparagraph*{(1)} $2\cup 3=\{0,1\}\cup \{0,1,2\}=\{0,1,2\}=3$;

\subparagraph*{(2)} $2\cap 3=\{0,1\}\cap \{0,1,2\}=\{0,1\}=2$;

\subparagraph*{(3)} $\cup 5=\cup \{0,1,2,3,4\}=4$;

\subparagraph*{(4)} $\cap 6=\cap \{0,1,2,3,4,5\}=0$.

\subparagraph*{(5)} $\cup\cup 7=\cup 6=5$.

\paragraph*{3} 证明: 考虑数学归纳法. 对于 $n=1$, 有 $1=0^+$;
对于一般的 $n$, 有 $n^+$ 是 $n$ 的后继. 由数学归纳法知结论成立. \QED

\paragraph*{5} 证明: 任取 $b\in B, B\in A^+$, 由 $A^+=A\cup \{A\}$,
知 $B\in A$ 或者 $B=A$. 若 $B=A$, 那么已有 $b\in A$;
若 $B\in A$, 那么由 $A$ 是传递集, 知 $b\in A$.
又 $A\subseteq A^+$, 故 $b\in A^+$, 因此 $A^+$ 是传递集. \QED

\paragraph*{7} 证明: 设 $n, n_0 \in \mathbb{N}$, 且 $h(n)=h(n_0)$, 
不妨设 $n=n_0$ 或 $n< n_0$. 下证: $n=n_0$.

对 $n$ 使用数学归纳法.
当 $n=0$ 时, 由 $h(n_0)=h(0)=a$, 若 $n_0>0$, 记 $n_0=m_0^+$, 
则 $h(n_0)=f(h(m_0))\in \ran f$, 与 $a\notin \ran f$ 矛盾. 
因此 $n_0=0=n$.

当 $n>0$ 时, 记 $n=m^+, n_0=m_0^+$, 由 $h(n)=h(n_0)$, 
及 $h(n)=f(h(m))$, 知 $f(h(m))=f(h(m_0))$, 
根据 $f$ 是单射, 知 $h(m)=h(m_0)$, 根据归纳假设, 得 $m=m_0$,
因此 $n=n_0$. \QED

\section*{习题五}

\paragraph*{2} 证明: 

\subparagraph*{(1)} 当 $f=g$ 时, $\ran f=\ran g$,
因此 $R$ 是自反的.

当 $fRg$ 且 $gRh$ 时, $\ran f=\ran g=\ran h$. 故 $fRh$,
因此 $R$ 是传递的. 

当 $fRg$ 时, $\ran f=\ran g$, 因此 $fRg$ 且 $gRf$,
因此 $R$ 是对称的. \QED

\subparagraph*{(2)} 构造映射 $\varphi: (A\mathop{\to} A)/R \mathrel{\to} P(A)-\{\varnothing\}$,
$\varphi(F)=\ran f, \textrm{where } f\in F$.
下证: $\varphi$ 是双射.

对于 $F, G\in (A\mathop{\to} A)/R$, 若 $\varphi(F)=\varphi(G)$, 
那么 $\forall f\in F, g\in G, \ran f = \ran g$, 那么 $\pair{f,g} \in R$,
故 $F=G$. 因此 $\varphi$ 是单射.

任取 $S \in P(A)-\{\varnothing\}$, 可构造 $f:A\to A$ 使 $\ran f=S$
(例如取 $s\in S$,令 $\forall a\in S, f(a)=a$,
$\forall a \in A-S, f(a)=s$), 那么对于 $f\in F\in (A\mathop{\to} A)/R$,
有 $\varphi(F)=S$, 因此 $\varphi$ 是满射. \QED

\paragraph*{5} 证明: 对 $n$ 使用数学归纳法.

$n=0$ 时结论平凡. $n=1$ 时, 若 $\epsilon$ 是 $n$ 的真子集, 只有 $\epsilon=0$, 
因此 $\epsilon \approx 0 \in 1$.

当 $n>1$ 时, 记$\epsilon$ 是 $n$ 的真子集, $n=m^+$. 若 $m\notin \epsilon$, 
那么由归纳假设, 存在 $p\in m\subseteq n$ 使 $\epsilon \approx p$.
若 $m\in \epsilon$, 那么 $\epsilon-\{m\}$ 是 $m$ 的真子集,
由归纳假设, 存在 $p_0\in m$ 使 $\epsilon-\{m\} \approx p_0$,
此时 $p=p_0^+$ 满足 $\epsilon \approx p$, 且 $p\in n$.

综上, 存在 $p\in n$ 使得$\epsilon \approx p$. \QED

\paragraph*{11} 

\subparagraph*{(1)} 可构造 $f:\mathbb{N}\to A, f(n)=n^7$,
$g:A\to\mathbb{N}, g(n)=n$, 因此 $\aleph_0 \le \card A \le \aleph_0$,
故 $\card A = \aleph_0$;

\subparagraph*{(2)} 同 (1) 知 $\card B = \aleph_0$;

\subparagraph*{(3)} $\card(A\cup B)=\card A + \card B = \aleph_0$;

\subparagraph*{(4)} $\card(A\cap B)=\card \{n^{7\times 109}\mid n\in \mathbb{N} \conj n\ne 0\} = \aleph_0$;

\paragraph*{12} 证明: 由 $A\approx B$, 存在双射 $f: A\to B$. 
构造 $g: P(A) \to P(B)$, 使 $\forall S \subseteq A, g(S)=f(S)$.
下证: $g$ 是双射.

设 $S,S'\subseteq A$, 若 $g(S)=g(S')$, 那么 $\forall x\in S$, 
有 $f(x)\in g(S) = g(S')$, 因此 $x = f^{-1}(f(x)) \in S'$,
故 $S\subseteq S'$, 同理 $S'\subseteq S$, 因此 $S=S'$, 故 $g$ 是单射.

任取 $T\subseteq B$, 令 $S=f^{-1}(T) \subseteq A$, 那么 $g(S)=f(S)=T$,
因此 $g$ 是满射. \QED

\paragraph*{补充题4} 

\subparagraph*{(1)} 记 $\card A=\kappa$, 由 $A\to\varnothing=\varnothing$,
知 $\kappa^0=0$.

\subparagraph*{(2)} 记 $\card A=\kappa$, 由 $\varnothing\to A=\varnothing$,
知 $0^\kappa=0$.

\subparagraph*{(3)} 记 $\card A = \card B =\kappa$, 且 $A\cap B=\varnothing$,
那么 $\kappa+\kappa=\card(A\cup B)$, $2\kappa=\card(2\times A)$.
由 $A\approx B$, 存在双射 $\varphi: A\to B$.
构造 $f:2\times A \to A\cup B, f(\pair{0, a})=a, f(\pair{1, a})=\varphi(a)$.
下证: $f$ 是双射.

设 $\pair{n, a}, \pair{n',a'}\in 2\times A$, 若 $f(\pair{n, a})=f(\pair{n',a'})$,
必有 $n=n'$, 否则不妨设 $n=0, n'=1$, 那么 $f(\pair{n,a})\in A, f(\pair{n',a'})\in B$,
与 $A\cap B=\varnothing$ 矛盾. 因此 $n=n'$, 故 $a=a'$ 或 $\varphi(a)=\varphi(a')$,
后者由于 $\varphi$ 是双射, 同样可推出 $a=a'$, 因此 $f$ 是单射.

对任意 $b\in A\cup B$, 若 $b\in A$, 有 $f(\pair{0, b})=b$, 否则 $b\in B$,
有 $f(\pair{1, \varphi^{-1}(b)})=b$, 因此 $f$ 是满射. \QED


% 第5章习题:11、12,补充题4
\end{document}
