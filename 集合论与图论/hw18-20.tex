\documentclass{ctexart}
\usepackage{multicol}
\usepackage{amsmath, amsfonts, amssymb}
\usepackage{tikz, tkz-graph}
\usepackage{pgffor}
\usepackage{pifont}
\usepackage{nicematrix}
\usepackage{subfigure}
\usepackage{geometry}
\usepackage{IEEEtrantools}
\special{dvipdfmx:config z 0} % delete this when release

\usetikzlibrary{positioning}

\geometry{a4paper,scale=0.8}

\title{集合论与图论~作业18-20}
\author{庄嘉毅}
\date{December 2022}

\def\QED{\hfill $\square$}
\def\st{\textrm{s.t.}\,}
\def\pair#1{\left\langle #1 \right\rangle}
\def\conj{\mathrel{\wedge}}
\def\disj{\mathrel{\vee}}
\def\equ{\mathrel{\Leftrightarrow}}
\def\restr{\mathbin{\upharpoonright}}
\def\ple{\mathrel{\preccurlyeq}}
\DeclareMathOperator{\dom}{dom}
\DeclareMathOperator{\ran}{ran}
\DeclareMathOperator{\fld}{fld}
\DeclareMathOperator{\card}{card}
\DeclareMathOperator{\e}{e}
\newcommand\tkzinline[2][90]{
    \mathord{
        \,\begin{tikzpicture}[rotate=#1, baseline=(current bounding box.center)]
            #2
        \end{tikzpicture}\,
    }
}


\everymath{\displaystyle}
% \linespread{2}

\begin{document}

\maketitle

\section*{习题十二}

\paragraph*{1}
\subparagraph*{(1)}
\begin{IEEEeqnarray*}{rCl}
    \tkzinline{
        \GraphInit[vstyle=Hasse]
        \Vertices[unit=1]{circle}{1,2,3,4,5}
        \Edges(1,2,3,5,2)
        \Edges(4,5)
    } & = &\tkzinline{
        \GraphInit[vstyle=Hasse]
        \Vertices[unit=1]{circle}{1,2,3,4,5}
        \Edges(1,2,3,5,2)
        \Edges(3,4,5)
    } + \tkzinline{
        \GraphInit[vstyle=Hasse]
        \Vertices[unit=1]{circle}{1,2,3,4}
        \Edges(1,2,3,4,2)
    }                 \\
    & =& \tkzinline{
        \GraphInit[vstyle=Hasse]
        \Vertices[unit=1]{circle}{1,2,3,4,5}
        \Edges(1,2,3,5,2,4)
        \Edges(3,4,5)
    } + 2\tkzinline{
        \GraphInit[vstyle=Hasse]
        \Vertices[unit=1]{circle}{1,2,3,5}
        \Edges(1,2,3,5,2)
    }                 \\
    & = &\tkzinline{
        \GraphInit[vstyle=Hasse]
        \Vertices[unit=1]{circle}{1,2,3,4,5}
        \Edges(1,2,3,5,2,4)
        \Edges(3,4,5,1)
    } + \tkzinline{
        \GraphInit[vstyle=Hasse]
        \Vertices[unit=1]{circle}{1,2,3,4}
        \Edges(1,2,3,1)
        \Edges(3,4,1)
        \Edges(2,4)
    } + 2\left(
    \tkzinline{
        \GraphInit[vstyle=Hasse]
        \Vertices[unit=1]{circle}{1,2,3,4}
        \Edges(4,1,2,3,4,2)
    } + \tkzinline{
        \GraphInit[vstyle=Hasse]
        \Vertices[unit=1]{circle}{1,2,3}
        \Edges(1,2,3,1)
    }
    \right) \\
    &=& \tkzinline{
        \GraphInit[vstyle=Hasse]
        \Vertices[unit=1]{circle}{1,2,3,4,5}
        \Edges(1,2,3,5,2,4)
        \Edges(3,4,5,1,3)
    } + 4\tkzinline{
        \GraphInit[vstyle=Hasse]
        \Vertices[unit=1]{circle}{1,2,3,4}
        \Edges(1,2,3,1)
        \Edges(3,4,1)
        \Edges(2,4)
    } + 4\tkzinline{
        \GraphInit[vstyle=Hasse]
        \Vertices[unit=1]{circle}{1,2,3}
        \Edges(1,2,3,1)
    } \\
    &=& \tkzinline{
        \GraphInit[vstyle=Hasse]
        \Vertices[unit=1]{circle}{1,2,3,4,5}
        \Edges(1,2,3,5,2,4,1)
        \Edges(3,4,5,1,3)
    } + 5\tkzinline{
        \GraphInit[vstyle=Hasse]
        \Vertices[unit=1]{circle}{1,2,3,4}
        \Edges(1,2,3,1)
        \Edges(3,4,1)
        \Edges(2,4)
    } + 4\tkzinline{
        \GraphInit[vstyle=Hasse]
        \Vertices[unit=1]{circle}{1,2,3}
        \Edges(1,2,3,1)
    }
\end{IEEEeqnarray*}

因此
\begin{IEEEeqnarray*}{rCl}
    f(G,k) &=& f(K_5,k) + 5f(K_4, k) + 2f(K_3, k) \\
    &=& k(k-1)(k-2)(k-3)(k-4) + 5k(k-1)(k-2)(k-3) + 4k(k-1)(k-2) \\
    &=& k^5-5k^4+9k^3-7k^2+2k
\end{IEEEeqnarray*}

\subparagraph*{(2)} $\chi(G)=\min\{5,4,3\}=3$.

\subparagraph*{(3)} $f(G, \chi(G))=f(G, 3)=4\times 3! =24$.

$f(G,4)=5\times 4! + 4\times 4!=216$.

\paragraph*{2}

$\tkzinline{
        \GraphInit[vstyle=Normal]
        \Vertices[unit=1.2]{circle}{1,2,3,4,5}
        \Edges(1,2,3,5,2)
        \Edges(4,5)
    }$ 如图, 记点割集 $V_1=\{v_2,v_5\}$.

\medskip

那么 $G-V_1$的联通分支分别为 $G_1=G[v_1],G_2=G[v_3],G_3=G[v_4]$.

\medskip

对应的有 $H_1=\tkzinline{
        \GraphInit[vstyle=Normal]
        \Vertices[unit=1]{circle}{1,2,5}
        \Edges(1,2,5)
    }, H_2=\tkzinline{
        \GraphInit[vstyle=Normal]
        \Vertices[unit=1]{circle}{2,3,5}
        \Edges(2,3,5,2)
    }, H_3=\tkzinline{
        \GraphInit[vstyle=Normal]
        \Vertices[unit=1]{circle}{2,4,5}
        \Edges(2,5,4)
    }$,

\medskip

故

\begin{IEEEeqnarray*}{rCl}
    f(G,k) &=& \frac{\displaystyle\prod_{i=1}^{3} f(H_i, k)}{f^2(G[V_1], k)} \\
    &=& \frac{f(K_3,k)\left( f(K_3,k) + f(K_2, k) \right)^2}{f^2(K_2, k)}\\
    &=& \frac{k(k-1)(k-2)\left(k(k-1)(k-2)+k(k-1)\right)^2}{k^2(k-1)^2}\\
    &=& k(k-1)(k-2)(k-2+1)^2\\
    &=& k^5-5k^4+9k^3-7k^2+2k
\end{IEEEeqnarray*}

\paragraph*{3} 记 $G=T_n \cup L_m$, 其中 $T_n$ 为树, $L_m$ 为圈, 则

\begin{IEEEeqnarray*}{rCl}
    f(G,k) &=& f(T_n,k) \cdot f(L_m, k) \\
    &=& k(k-1)^{n-1}\left((k-1)^m + (-1)^m(k-1)\right)\\
    &=& k(-1)^m(k-1)^n+k(k-1)^{m+n-1}
\end{IEEEeqnarray*}

\paragraph*{11} 记 $G$ 是一个 $n$ 阶3-正则哈密顿图,
$G$ 的一个哈密顿回路为 $L$.

考虑子图 $G-L$, 由于 $L$ 为2-正则图, 因此 $G-L$ 为1-正则图,
即 $G-L$ 为一系列 $K_2$ 的不交并. 同时可知 $L$ 为偶圈.

因此可给出如下的3-染色构造: 将 $G-L$ 中的每条边都染成红色,
$L$ 中的边交替染为蓝色和白色, 使得相邻的边颜色不同.
此时, $G$的每个顶点都恰好与红色,蓝色,白色各一条边相邻,
所以 $G$ 是3-边色图.

\paragraph*{12}

\subparagraph*{(1)} 考虑如下的彼得森图:

\medskip

$\tkzinline{
        \GraphInit[vstyle=Normal]
        \Vertices[unit=2.2]{circle}{1,2,3,4,5}
        \Vertices[unit=1]{circle}{6,7,8,9,10}
        \Edges(1,2,3,4,5,1)
        \Edges(6,8,10,7,9,6)
        \Edges(1,6)
        \Edges(2,7)
        \Edges(3,8)
        \Edges(4,9)
        \Edges(5,10)
    }$

\medskip

由 Vizing 定理, $\chi^r(G)=3$ 或 $\chi^r(G)=4$.

假设 $\chi^r(G)=3$, 考虑外围圈 $L_1=\{v_1,v_2,v_3,v_4,v_5\}$ 的染色情况,
讨论可知, 非同构的边染色只有一种, 不妨令 $(v_1,v_2), (v_3,v_4)$ 为红色,
$(v_2,v_3), (v_4,v_5)$ 为蓝色, $(v_5,v_1)$ 为绿色.

那么边 $(v_2,v_7)$ 与边 $(v_3,v_8)$ 只能染绿色,
边 $(v_5,v_{10})$ 只能染红色. 如图:

\medskip

$\tkzinline{
        \GraphInit[vstyle=Normal]
        \Vertices[unit=2.2]{circle}{1,2,3,4,5}
        \Vertices[unit=1]{circle}{6,7,8,9,10}
        \Edges(6,8,10,7,9,6)
        \Edges(1,6)
        \Edges(4,9)
        \SetUpEdge[color=red]
        \Edges(1,2)\Edges(3,4)
        \Edges(5,10)
        \SetUpEdge[color=blue]
        \Edges(2,3)\Edges(4,5)
        \SetUpEdge[color=teal]
        \Edges(5,1)
        \Edges(2,7)
        \Edges(3,8)
    }$

\medskip

此时考虑 $(v_7,v_{10})$ 与 $(v_8,v_{10})$ 的染色情况, 可知这两条边只能染蓝色,
而这两条边都和 $v_{10}$ 相邻, 矛盾.

故不存在三种颜色的染色方法, 因此 $\chi^r(G)=4$.

\subparagraph*{(2)} 由上一题知, 3-正则哈密顿图的边染色数为3.
而彼得森图是3-正则图, 但边染色数为4, 因此彼得森图不是哈密顿图.


\paragraph*{13} 证明: 易知满足条件的图至少有2个顶点且不是零图.

对于连通的无环的平面图 $G$, 无奇圈等价于 $G$为二部图,
进而等价于$G$可2-着色, 又因为$G$不是零图, 因此这等价于 $G$ 的着色数为2.

考虑 $G$ 的对偶图 $G^*$, 则$G$是欧拉图等价于 $G$ 的顶点度数均为偶数,
进而等价于 $G^*$ 的面的边数均为偶数, 这等价于 $G^*$ 无奇圈,
因此等价于 $G^*$ 是2-着色的, 即 $G$ 是2-面着色的. \QED

\newpage

\section*{习题十三}

\paragraph*{1}

\subparagraph*{(1)} 极小支配集有 $\{v_1,v_3\}, \{v_1,v_2\},
    \{v_1,v_4\}, \{v_1,v_5\}, \{v_2,v_3\}, \{v_3,v_4\},
    \{v_3,v_5\}, \{v_2,v_4,v_5\}$.

支配数 $\gamma_0=2$.

\subparagraph*{(2)} 极小点覆盖集有 $\{v_1,v_3\}, \{v_2,v_4,v_5\}$.

点覆盖数 $\alpha_0=2$.

\subparagraph*{(3)} 极大点独立集有 $\{v_1,v_3\}, \{v_2,v_4,v_5\}$.

点独立数 $\beta_0=3$.

\subparagraph*{(4)} 匹配有 $\{a,c\}, \{a,f\}, \{b,d\}, \{b,f\},
    \{e,c\}, \{e,f\}$.

匹配数 $\beta_1=2$.


\subparagraph*{(5)} 极小边覆盖有 $\{a,b,e\}, \{a,b,f\}, \{a,c,e\},
    \{a,c,f\}, \{d,b,e\}, \{d,b,f\}, \{d,c,e\}, \{d,c,f\}$.

边覆盖数 $\alpha_1=3$.

\paragraph*{3} 证明: 任取 $G$ 的一个点覆盖集 $V_1$, 若 $V_1$ 包含 $G$
全部顶点, 那么易知 $|V_1| > \delta(G)$. 若存在 $V_1$ 以外的顶点 $v_0$,
那么与 $v_0$ 关联的所有边一定与 $V_1$ 中的某个顶点关联, 记 $v_0$ 的度数为 $d$,
有 $|V_1| \ge d \ge \delta(G)$.

综上, 由 $V_1$ 的任意性, 可知 $\alpha_0 \ge \delta(G)$. \QED

\paragraph*{5} 证明: 首先证明, 当 $G$ 存在完美匹配时, 该游戏有后手必胜策略.

一个后手必胜策略如下: 取 $G$ 的一个完美匹配 $M$, 之后无论先手方如何选取顶点 $v$,
后手方总可以选择 $v$ 被 $M$ 匹配的顶点 $v'$, 并且由匹配的性质, 这样的 $v'$
不会被之前选取过. 按照这一策略, 双方最终会取尽 $G$ 中所有顶点,
并且最后一次选取由后手方做出, 因此后手方必胜.

下面证明, 当 $G$ 不存在完美匹配时, 该游戏有先手必胜策略.

此时, 取 $G$ 的一个最大匹配 $M$, 先手第一步选取 $M$ 的一个非饱和点 $v_0$.
由 $M$ 的极大性, 与 $v_0$ 相邻的所有顶点都是 $M$ 的饱和点,
因此后手方第二步的选择 $v_1$ 只能是 $M$ 的饱和点.

之后, 在导出子图 $G'=G[M]$ 内, $M$ 是 $G'$ 的一个完美匹配,
先手方可以使用上述的后手必胜策略. 注意到按照这个策略, 后手方选择的点必在 $G'$ 内,
否则双方选择的点连成 $M$ 的一条可增广路径, 与 $M$ 的最大性矛盾.
因此先手方使用这个策略, 可以使后续能选取到的顶点都在 $G'$ 内,
从而在 $G'$ 内选尽所有顶点获胜.

\paragraph*{8} 将同学与课外小组视作顶点, 成员关系视作边,
则选组长可以转化为寻找二部图中的完备匹配.

记三个小组的成员分别为 $A, B, C$, 那么根据穷举法, 完备匹配的个数为:
\begin{align*}
    N & = \sum_{a\in A} \sum_{b\in B-\{a\}} \left| C-\{a,b\} \right|
\end{align*}

当 $N>0$ 时, 可判定存在完备匹配.

\subparagraph*{(1)} $\tkzinline[0]{
        \GraphInit[vstyle=Hasse]
        \Vertices[x=1,y=2,unit=1]{line}{1,2,3}
        \Vertices[x=0,y=0,unit=1]{line}{4,5,6,7,8}
        \Edges(1,4,2,6,3,8,2,4)
        \Edges(1,5)
        \Edges(3,7)
    }$
此时共有 11 种选组长的方法.

\subparagraph*{(2)} $\tkzinline[0]{
        \GraphInit[vstyle=Hasse]
        \Vertices[x=1,y=2,unit=1]{line}{1,2,3}
        \Vertices[x=0,y=0,unit=1]{line}{4,5,6,7,8}
        \Edges(1,4)
        \Edges(2,5,3,6,2,8,3,7)
    }$
此时共有 9 种选组长的方法.

\subparagraph*{(3)} $\tkzinline[0]{
        \GraphInit[vstyle=Hasse]
        \Vertices[x=1,y=2,unit=1]{line}{1,2,3}
        \Vertices[x=0,y=0,unit=1]{line}{4,5,6,7,8}
        \Edges(1,4,2)
        \Edges(5,3,6)
        \Edges(7,3,8)
    }$
此时不存在选组长的方法.

\end{document}

