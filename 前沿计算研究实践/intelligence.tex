\documentclass{ctexart}
\usepackage{multicol}
\usepackage{amsmath, amsfonts, amssymb}
\usepackage{geometry}
\special{dvipdfmx:config z 0} % delete this when release

\geometry{a4paper,scale=0.8}

\title{前沿计算研究实践~概念报告~智能}
\author{庄嘉毅}
\date{November 2022}

\def\QED{\hfill $\square$}
\def\st{\textrm{s.t.}\,}
\def\pair#1{\left\langle #1 \right\rangle}
\def\conj{\mathrel{\wedge}}
\def\disj{\mathrel{\vee}}
\def\equ{\mathrel{\Leftrightarrow}}
\def\restr{\mathbin{\upharpoonright}}
\def\ple{\mathrel{\preccurlyeq}}
\DeclareMathOperator{\dom}{dom}
\DeclareMathOperator{\ran}{ran}
\DeclareMathOperator{\fld}{fld}
\DeclareMathOperator{\card}{card}
\DeclareMathOperator{\e}{e}

\everymath{\displaystyle}
% \linespread{2}

\begin{document}

\maketitle

何为``人工智能''?这是一个难以回答的问题。从被称为``人工智能起源''的达特茅斯会议之后,
人工智能一直是非常模糊、充满争议的概念。争议的焦点集中于二:
什么是智能?什么样的程序可以被称为``人工智能''?

符号主义、连接主义、行为主义分别从数理逻辑、仿生学、控制论的角度解释人工智能,
各自催生了一系列的人工智能理论,带来了机器推理、神经网络、智能机器人等一系列的技术。
今日,这几种声音间的界限已经不在清晰,人工智能的研究已经不再局限于某一种理论,
而是在不断地融合、发展。

那么,在人工智能已经走入大众视野的今天,我们可以用怎样的方式界定人工智能呢?

从后经验主义的视角,人工智能相对于``非人工智能''而言。
计算机程序可以视为对给定任务的一种建模,其智能与否取决于其完成的任务的性质。
比如说,一个排序算法从来不会被认为是智能的,因为排序这个任务是机械的。
而一个能够自动完成数学证明的程序则被认为是智能的,因为数学证明是一种创造性的任务。
一个图像分类程序也可以被认为是智能的,因为图像分类是需要学习才能完成的。
但一个网络拥塞的动态控制算法则不被看做智能的,尽管它内部有许多可以被学习的参数,
但整体上来说,它的任务是不变的。

从这个角度来看,人工智能是一种人工设计的、能够完成智能体完成的任务的系统。
其最终的结果,是代替智能的主体——人类,完成一些任务。而其过程来源于对任务的建模方式。
这种建模方式可以是数学的,也可以是仿生的,也可以是行为的。
但殊途同归,最终的结构都是一样的:一个能够完成任务的系统。

\end{document}
