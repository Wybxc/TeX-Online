\documentclass{ctexart}
\usepackage{multicol}
\usepackage{amsmath, amsfonts, amssymb}
\usepackage{geometry}
\special{dvipdfmx:config z 0} % delete this when release

\geometry{a4paper,scale=0.8}

\title{前沿计算研究实践~概念报告~信息}
\author{庄嘉毅}
\date{October 2022}

\def\QED{\hfill $\square$}
\def\st{\textrm{s.t.}\,}
\def\pair#1{\left\langle #1 \right\rangle}
\def\conj{\mathrel{\wedge}}
\def\disj{\mathrel{\vee}}
\def\equ{\mathrel{\Leftrightarrow}}
\def\restr{\mathbin{\upharpoonright}}
\def\ple{\mathrel{\preccurlyeq}}
\DeclareMathOperator{\dom}{dom}
\DeclareMathOperator{\ran}{ran}
\DeclareMathOperator{\fld}{fld}
\DeclareMathOperator{\card}{card}
\DeclareMathOperator{\e}{e}

\everymath{\displaystyle}
% \linespread{2}

\begin{document}

\maketitle

信息的概念由来已久。语言作为信息的载体,人类语言的发明即意味着对信息概念的认知。
很长一段时间里,在科学与哲学的研究中,人们对信息的理解都是从经验主义的角度出发,
把信息当作一种固有的、不证自明的概念,而无需探究其本质。
直到第二次工业革命,无线电技术诞生,信息获得了新的载体,
首次成为可被把握的实在的存在;此时人们意识到信息本身也是可以,并且应该被认识的。

香农最早对信息给出一个理论自洽的完整的定义。香农把信息定义为
“用于消除不确定性的东西”,我们知道的信息越多,
越能在复杂的概率空间里确证确定性的存在。
由此,他提出了信息熵的数学形式:

\begin{equation*}
H(X) = -\sum_{x\in\dom(X)} p(x)\log_2 p(x)
\end{equation*}

时至今日,现代人谈论起信息,言必称香农的信息熵。这是一个值得思考的现象:
为何信息熵这个概念能够如此广泛地被接受,即使信息技术的发展早已超越香农的时代?

从实践的层面上看,香农的信息熵公式达成了对信息定量的表述,
给予人们将信息纳入其他工程领域的机会。信息熵描述了理论上信息的极限量,
“无信息”与“完全信息”可以先验地预测,使得信息传递的实践有迹可循。

从哲学的层面上看,香农的信息熵源于对信息的认识论理解,
也开启了通向本体论视角的道路。信息熵描述的是认识主体对信息的认识程度,
但其数值完全由认识客体的固有性质决定。
在二十世纪理性主义转向非理性主义的哲学浪潮中,本体论再次复苏,
信息熵得到再解释,将信息的理解带入新的高度。


\end{document}
