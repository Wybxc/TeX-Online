\documentclass[twocolumn]{ctexart}
\usepackage[backend=biber,style=gb7714-2015,hyperref=true,backref]{biblatex}
\usepackage{hyperref}
\usepackage{amsmath, amsfonts, amssymb, amsthm}
\usepackage{algorithm, algorithmicx, algpseudocode}
\usepackage{geometry}
\special{dvipdfmx:config z 0} % delete this when release

\geometry{a4paper,scale=0.8}

\title{图灵停机问题与软件静态检查研究}
\author{庄嘉毅}
\date{Decemeber 2022}

\def\QED{\hfill $\square$}
\def\st{\textrm{s.t.}\,}
\def\pair#1{\left\langle #1 \right\rangle}
\def\conj{\mathrel{\wedge}}
\def\disj{\mathrel{\vee}}
\def\equ{\mathrel{\Leftrightarrow}}
\def\restr{\mathbin{\upharpoonright}}
\def\ple{\mathrel{\preccurlyeq}}
\DeclareMathOperator{\dom}{dom}
\DeclareMathOperator{\ran}{ran}
\DeclareMathOperator{\fld}{fld}
\DeclareMathOperator{\card}{card}
\DeclareMathOperator{\e}{e}

\newtheorem{theorem}{定理}

\addbibresource{turing-halting-ref.bib}

\begin{document}

\maketitle

\begin{abstract}
    停机问题是一个计算理论中的不可判定问题。静态检查是用计算机程序对形式化约束进行逻辑推导,
    发现软件可能的问题与漏洞的方法。停机问题可看作广义的静态检查,
    因此可以将停机问题与静态检查器联系起来,用停机问题的一些结论指导静态检查器的设计。
    由于实际计算模型的计算能力弱于图灵机,对于真实的任务,
    考虑图灵机与实际计算机之间的差异,利用更多现实计算机的特性,
    借助更具体的信息执行有效的静态检查。
\end{abstract}

\section{引言}

形式化验证是软件工程中的一种重要方法,它的目标是通过数学方法证明软件的正确性。
其基本思想是将软件的行为抽象为数学模型,然后通过数学方法证明模型的正确性。

静态检查是一种较弱的形式化验证机制。对于待检查程序的一部分已知的约束,建立形式化验证的数学模型,之后用计算机程序进行逻辑推导,判断软件是否满足模型的正确性条件,
从而发现可能存在的问题与漏洞。许多静态检查工具已经在实践中得到广泛应用,
例如 TypeScript 的类型检查器\cite{typescript},以及 Rust 的生命周期检查器\cite{rustlifetimes}。

停机问题是计算理论中的一个重要问题,它的目标是判断一个图灵机是否会停机。
图灵在 1936 年指出,停机问题是一个不可判定的问题,即无法用图灵机程序判断任意一个图灵机是否会停机\cite{turing1936computable}。

停机问题中涉及到以图灵机程序为输入,给出逻辑判断的程序,
这一过程与静态检查器十分相似。因此,我们可以将停机问题与静态检查器联系起来,
并将停机问题的一些结论指导静态检查器的设计。

\section{停机问题}

\subsection{图灵机}

图灵机是一种抽象的计算模型,它由一个无限长的纸带和、
一个读写头和一个有限大小的状态寄存器组成。纸带上的每个单元都可以存储一个符号,
读写头可以读取或写入当前单元的符号,并且可以将读写头移动到纸带上的任意一个单元。

图灵机程序是一套控制规则,它规定了在特定状态和读取到特定符号时,
读写头的移动与读写操作,以及状态寄存器的更新。
由于状态寄存器的大小和符号种类是有限的,因此一个确定的图灵机程序具有有限的大小。

\subsection{停机问题}

图灵机的状态中包含一个特殊的停机状态,当图灵机进入停机状态时,它会停止运行。
停机问题的目标是判断一个图灵机程序是否会导致图灵机进入停机状态。

停机问题是不可判定问题,即不存在一个图灵机程序,
能够判断任意一个图灵机程序是否会停机。

通过简单的反证法,可以证明这一命题。

\begin{theorem}\label{thm:turing-halting-undecidable}
    停机问题是不可判定问题。
\end{theorem}

\begin{proof}
    假设存在一个图灵机程序,能够判断任意一个图灵机程序是否会停机,
    记这个程序为 $P$。可以将 $P$ 看作映射 $P: \mathcal{M} \times \mathcal{I} \to \{0, 1\}$,
    其中 $\mathcal{M}$ 是所有图灵机程序的集合,
    $\mathcal{I}$ 是图灵机输入的集合,
    $P(m, i)=1$ 表示图灵机程序 $m$ 在输入 $i$ 时会停机,
    反之,$P(m, i)=0$ 表示图灵机程序 $m$ 在输入 $i$ 时不会停机。

    另外,存在一个永不停机的图灵机程序 $F$,如无限循环的图灵机程序。

    接下来,构造如下的图灵机程序 $G$:

    \begin{equation*}
        G(m) := \begin{cases}
            0   & \text{if } P(m, m)=0, \\
            F() & \text{if } P(m, m)=1
        \end{cases}
    \end{equation*}

    考查 $G(G)$ 的停机情况。如果 $P(G,G)=1$,说明 $G(G)$ 会停机,
    由于 $G(G)=F()$,因此 $F$ 会停机,这与 $F$ 永不停机矛盾。
    如果 $P(G,G)=0$,说明 $G(G)$ 不会停机,而 $G(G)=0$,
    因此 $G$ 会停机,矛盾。

    因此,$G$ 无法判断 $F$ 是否会停机,与假设矛盾。
    故停机问题是不可判定问题。
\end{proof}

\subsection{停机问题的一个伪证}

下面的证明过程试图说明停机问题是可判定问题。

\begin{proof}[伪证]
    编写一个模拟图灵机的图灵机程序 $T$,它接受一个图灵机程序和一段输入,
    模拟图灵机运行。每次模拟一步,就保存一份当前状态和已写入纸带的内容,
    那么:

    \begin{enumerate}
        \item 如果图灵机进入停机状态,那么 $T$ 也会进入停机状态。
              此时可判断输入的图灵机程序可停机。

        \item 如果图灵机不会停机,那么将在有限步内进入循环,
              此时 $T$ 检测到出现与之前相同的状态和纸带内容,
              可判断输入的图灵机程序不会停机。
    \end{enumerate}

    于是,$T$ 可以判断任意一个图灵机程序是否会停机。
    因此停机问题是可判定问题。
\end{proof}

上述证明过程中,存在一个隐藏的漏洞。
它错误地假设不停机的图灵机程序一定会进入循环,事实上,
由于图灵机的纸带是无限长的,因此不停机的图灵机程序也可能不进入循环,
比如一直向纸带的一个方向移动。因此,通过检测重复状态的出现,
无法判断图灵机程序是否会停机。

这个伪证揭示了图灵机模型与现实存在的计算机间的差异。
现实的计算机总是拥有有限的内存空间,因此状态数是有限的,而图灵机的纸带是无限长的,
可以具有无限多的状态。这说明,图灵机的计算能力比现实的计算机更强。
而对于比图灵机更弱的计算模型,是存在判定停机的可能的。

\section{软件静态检查}

\subsection{类型检查}

类型检查是软件静态检查中基本而重要的一种检查方式。
类型检查的思想类似于物理学中的量纲检查\cite{typesystems},
通过检查运算的一部分信息,排除掉不符合要求的计算。

最初的类型系统设计是为了区分二进制数据的不同含义,比如整数、浮点数、内存地址等。
之后类型系统被不断扩展,能够处理更加复杂的数据结构,引入了参数化类型、
子类型等概念。随着类型系统的丰富,计算机科学家意识到类型系统与逻辑系统的联系,
即柯里-霍华德同构\cite{curry1934,curry1958,howard1980},
这为借助逻辑证明技术来验证类型系统提供了可能。

一个设计良好的类型系统和类型检查器可以发现程序中的很多错误,
例如内存别名、空指针、不合法参数等。这些检查主要是通过类型间的兼容性判断来实现的,
通过阻止不兼容类型的变量赋值与运算,避免程序中的错误。

随着类型系统的发展,类型逐渐脱离了二进制数据的解释方式这一含义,
类型可以是抽象的、不具有运行时含义的。例如,函数的签名可以包含对其中可能发生的异常的描述,类型检查器可以利用这些信息,在编译时检查程序中是否可能存在未处理的异常\cite{cpp,checkerr}。

Rust 语言的内存安全机制就是通过类型检查来实现的。通过将生命周期作为类型的一部分,
生命周期的比较通过子类型关系来判断,从而保证可访问的内存空间不会超出生命周期,
从而避免访问未初始化内存、访问已释放内存、内存竞争等错误\cite{rust}。

\subsection{停机问题与静态检查的局限}

停机问题可以看作一种广义的静态检查,这说明静态检查能够解决的问题是有限的。

一个例子是,对于通用图灵机,无法设计一个完美的内存泄漏检查器。

\begin{theorem}\label{thm:memoryleak}
    图灵机的内存泄漏检查是不可判定问题。
\end{theorem}

\begin{proof}
    考虑这样一个程序,它在运行后会创建两个子进程,
    其中一个子进程会等待另一个子进程的结束,
    等待期间,它会不断申请分配内存,并丢弃对这些内存的引用,
    直到等待到另一个子进程结束,进程退出,释放所有内存。

    那么,对这个程序是否存在内存泄漏,需要对它启动嗯怼子进程是否停机进行判定,
    但停机问题不可判定,因此对它的内存泄漏检查也是不可判定的。
\end{proof}

事实上,内存泄漏检查一直是软件工程的难题。即使是带有垃圾回收机制的语言,
也经常会出现内存泄漏的问题。Rust 语言曾经试图通过生命周期来解决内存泄漏,
但很快发现生命周期无法检测出所有的内存泄漏问题,
因此决定将内存泄漏排除在内存安全模型之外\cite{rustbug}。

\subsection{实际计算模型中的静态检查}

正如上文指出的,实际计算模型的计算能力弱于图灵机,
很多在图灵机上无法判定的问题,比如停机问题,在资源有限的计算模型中是可判定的。

另一方面,对于真实的任务,往往并不需要标准图灵机的计算能力,
静态检查器可以简化其检查标准,执行容忍度更低的检查,
而不会使程序的计算能力过多受限。许多领域特定语言被设计为非图灵完备的,
而它们依然得到了广泛的应用。

\section{总结}

静态检查是保证程序健壮性的重要手段,但由于图灵机模型的局限性,如停机问题,
完美的静态检查器是不可能存在的。但这并不意味着静态检查是不可能的任务,
因为实际的计算机的计算能力弱于图灵机,只需要更弱的检查,就可以保证程序的健壮性。

设计精巧、强大、高效的静态检查器,是软件工程的一个重要课题。
当静态检查的理论设计触及图灵机计算模型的上限时,
可以考虑图灵机与实际计算机之间的差异,利用更多现实计算机的特性,
借助更具体的信息执行有效的静态检查。

\printbibliography[heading=bibintoc,title=参考文献]

\end{document}
