\documentclass{ctexart}
\usepackage{multicol}
\usepackage{amsmath, amsfonts, amssymb}
\usepackage{geometry}
\special{dvipdfmx:config z 0} % delete this when release

\geometry{a4paper,scale=0.8}

\title{前沿计算研究实践~概念报告~计算}
\author{庄嘉毅}
\date{October 2022}

\def\QED{\hfill $\square$}
\def\st{\textrm{s.t.}\,}
\def\pair#1{\left\langle #1 \right\rangle}
\def\conj{\mathrel{\wedge}}
\def\disj{\mathrel{\vee}}
\def\equ{\mathrel{\Leftrightarrow}}
\def\restr{\mathbin{\upharpoonright}}
\def\ple{\mathrel{\preccurlyeq}}
\DeclareMathOperator{\dom}{dom}
\DeclareMathOperator{\ran}{ran}
\DeclareMathOperator{\fld}{fld}
\DeclareMathOperator{\card}{card}
\DeclareMathOperator{\e}{e}

\everymath{\displaystyle}
% \linespread{2}

\begin{document}

\maketitle

计算的发展史与数学相伴。对计算的需求是人类数学学科的起源。
后来,数学理论与计算技术开始分离,两者半独立而相互促进发展。
其中,西方走上了理论优先的发展历程,而中国走上了计算优先的发展历程。

计算技术的发展包括两个方面:算法和计算工具。计算工具的算力不断增强,
其对应的算法也变得复杂。算筹只是简单的计数工具,而算盘则可以用口诀加速计算,
对数的发现让计算尺成为工程师有力的工具。

计算工具的发展的极致是计算机。计算机改变了计算工具的概念,
使得计算工具的算力不再是人类的极限,计算工具走向自动化。
在此基础上,算法的重要性逐渐显现出来,计算工具的算力不再是算法的瓶颈,
而算法的复杂性反而成为计算工具的瓶颈。在传统的理解里,算法依附于数学理论而存在,
计算机的发明使得算法成为独立的学科,算法以数学理论与计算理论为基础,
在具体的工程实践中不断发展。



\end{document}
