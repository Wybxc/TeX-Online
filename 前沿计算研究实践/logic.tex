\documentclass{ctexart}
\usepackage{multicol}
\usepackage{amsmath, amsfonts, amssymb}
\usepackage{geometry}
\special{dvipdfmx:config z 0} % delete this when release

\geometry{a4paper,scale=0.8}

\title{前沿计算研究实践~概念报告~逻辑}
\author{庄嘉毅}
\date{October 2022}

\def\QED{\hfill $\square$}
\def\st{\textrm{s.t.}\,}
\def\pair#1{\left\langle #1 \right\rangle}
\def\conj{\mathrel{\wedge}}
\def\disj{\mathrel{\vee}}
\def\equ{\mathrel{\Leftrightarrow}}
\def\restr{\mathbin{\upharpoonright}}
\def\ple{\mathrel{\preccurlyeq}}
\DeclareMathOperator{\dom}{dom}
\DeclareMathOperator{\ran}{ran}
\DeclareMathOperator{\fld}{fld}
\DeclareMathOperator{\card}{card}
\DeclareMathOperator{\e}{e}

\everymath{\displaystyle}
% \linespread{2}

\begin{document}

\maketitle

对逻辑的理解,可以分为多个层面。在日常生活中,我们运用因果推理,运用排除和假设,
对未知的事情加以推测。在数学中,书写证明要考虑周密,避免逻辑错误。
在计算机科学中,我们运用逻辑来描述程序的行为,以及程序的正确性。
总体上来说,可以把逻辑分为两类:直觉的、非形式的朴素逻辑,以及可符号化的、形式的逻辑。

朴素逻辑直接源于人类对周遭事物的认识,从所见事物中提取出命题,
以经验的方式判断命题之间的关系。在朴素逻辑的观点中,逻辑是后验的,
即逻辑是由人类的认知过程产生的,而非逻辑产生认知。

形式逻辑诞生于数学符号化统一的过程中。数学家试图用先验的公理体系来描述数学世界,
在公理的基础上,通过推理来证明新的命题。这种推理的过程,需要一种先验的规定,
即先验的逻辑。

公理体系的一个重要特征是相容性,这可以由不同公理间的独立性来保证。
形式逻辑要求逻辑的符号化,势必需要找出一套最小的逻辑原理,
通过它们导出一系列在朴素逻辑中成立的规则。但问题在于,公理的推导依赖于逻辑,
而逻辑的推导却不能依赖逻辑自身,必须要假设一种先逻辑存在的规则,
才能让逻辑适用其中。

最终,逻辑学家将目光看向语言。把逻辑公式视作符号串,再预设一系列语法规则,
对逻辑符号进行展开或规约。这成功地构建了一种先逻辑的规则,让我们能够构建逻辑本身。

但回过头来看,形式逻辑的这种构建方式在自洽性上是成功的,
但其有用性依然需要经过朴素逻辑的检验。逻辑的形式化没有超越经验,
只是在以另一种方式表达经验。

这似乎与思辨哲学的发展历程有内在的一致性。从经验主义到理性主义,
再到非理性的后经验主义,我们的思辨方式在不断地变化,但只有一点是不变的,
我们所做的一切都是在认识我们生活的世界。

\end{document}
