\documentclass{ctexart}
\usepackage{amsmath, amsfonts, amssymb}
\usepackage{geometry}
\usepackage{algorithm, algorithmicx, algpseudocode}
% \special{dvipdfmx:config z 0} % delete this when release

\geometry{a4paper,scale=0.8}

\title{算法设计与分析~作业1}
\author{庄嘉毅}
\date{Feburary 2023}

\def\QED{\hfill $\square$}
\def\st{\textrm{s.t.}\,}
\DeclareMathOperator{\e}{e}

\algsetblockdefx[Function]{Function}{EndFunction}{}{}[3]{\textbf{function} \textsc{#1}(#2) \textbf{returns} #3}{\textbf{end function}}
\algsetblockdefx[Persistent]{Persistent}{EndPersistent}{}{}{\textbf{persistent:}}{}
\algsetblockdefx[For]{For}{EndFor}{}{}[1]{\textbf{for} #1 \textbf{do}}{\textbf{end for}}
\algsetblockdefx[If]{If}{EndIf}{}{}[1]{\textbf{if} #1 \textbf{then}}{\textbf{end if}}
\newcommand{\Let}[1]{\State #1 $\gets$}
\newcommand{\F}[2]{\textsc{#1}(#2)}
\newcommand{\Ret}[1]{\State \textbf{return} #1}

\begin{document}

\maketitle

\section{Textbook Exercises}

\paragraph*{1.18} 对下列函数, 按照他们的阶从高到低排列:
\begin{gather*}
    n!, 2^{2n}, n2^n, (\log n)^{\log n}=\Theta(n^{\log\log n}), n^3, \log n!=\Theta(n\log n), \\
    n=\Theta(\log 10^n), 2^{\sqrt{2\log n}}, 2^{\log \sqrt{n}}, \sum_{k-1}^n \frac{1}{k}=\Theta(\log n), \log\log n
\end{gather*}

\paragraph*{1.19} 求解以下递推方程:

\subparagraph*{(1)} \begin{align*}
    T(n) & = T(n-1) + n^2           \\
         & = \sum_{k=1}^n k^2       \\
         & = \frac{n(n+1)(2n+1)}{6} \\
         & = \Theta(n^3)
\end{align*}

\subparagraph*{(3)} \begin{align*}
    T(n) & = T(n/2) + T(n/4) + cn
\end{align*}

故
\begin{align*}
    T(n) - 4cn & = T(n/2) - 2cn + T(n/4) - cn                  \\
               & = F_{\log n}(T(2)-4c) + F_{\log n -1}(T(1)-c) \\
               & = \Theta(\varphi^{\log n})                    \\
               & = o(n)
\end{align*}
其中 $F_n$ 为 Fibonacci 数列, $\varphi=(1+\sqrt{5})/2 $.

因此
\begin{align*}
    T(n) & = 4cn + o(n) \\
         & =\Theta(n)
\end{align*}

\subparagraph*{(5)} \begin{align*}
    T(n) & = 5T(n/2) + (n\log n)^2                                      \\
         & = \sum_{k=1}^{\log n} 5^k \frac{n}{2^k} \log \frac{n}{2^k}   \\
         & = n \sum_{k=1}^{\log n} (\log n-k)\left(\frac{5}{2}\right)^k \\
         & = n \cdot \Theta\left((5/2)^{\log n}\right)                  \\
         & = n \cdot \Theta\left(n^{\log 5-1}\right)                    \\
         & = \Theta(n^{\log 5})
\end{align*}

\subparagraph*{(7)} \begin{align*}
    T(n) & = T(n-1) + \frac{1}{n}     \\
         & = \sum_{k=1}^n \frac{1}{k} \\
         & = \Theta(\log n)
\end{align*}

\paragraph*{1.21} 算法A: $T(n)=5T(n/2) + cn$, 由主定理可得 $T(n)=\Theta(n^{\log_2 5})$.

算法 B: $T(n)=2T(n-1) + c$, 解得 $T(n)=\Theta(2^n)$.

算法 C: $T(n)=9T(n/3) + O(n^3)$, 在最坏情况下, $T(n)=9T(n/3) + \Theta(n^3)$, 由主定理可得 $T(n)=\Theta(n^3)$.

因此在最坏情况下, 时间复杂度最低的算法为算法 A.

\newpage
\section{Partial Sum of a 1D Array}

\paragraph*{(a)} $T(n)=O(n^3)$.

\paragraph*{(b)} 固定 $i, j$, 计算 \texttt{A[i] + ... + A[j]} 的时间复杂度为 $\Theta(j-i)$, 将结果写入 \texttt{B[i, j]} 的用时为 $\Theta(1)$, 因此总时间复杂度为
\begin{align*}
    T(n) & = \sum_{i=1}^n \sum_{j=i+1}^n \left(\Theta(j-i) + \Theta(1)\right) \\
         & = \sum_{i=1}^n \sum_{j=i+1}^n \Theta(j-i)                          \\
         & = \sum_{i=1}^n \sum_{k=1}^{n-i} \Theta(k)                          \\
         & = \sum_{i=1}^n \Theta((n-i)^2)                                     \\
         & = \Theta(n^3)
\end{align*}

\paragraph*{(c)} 以下是时间复杂度为 $\Theta(n^2)$ 的算法:

\begin{algorithm}[ht]
    \caption{Partial Sum of a 1D Array}
    \begin{algorithmic}
        \Function{Partial-Sum}{$A$}{$B$}
        \Let{$n$} \F{Length}{$A$}
        \Let{$B$} \F{New-Array}{$n \times n$}
        \\
        \For{$i$ \textbf{in} $1,2,\ldots, n$}
            \Let{$B[i, i]$} $A[i]$
            \For{$j$ \textbf{in} $i+1,i+2,\ldots, n$}
                \Let{$B[i, j]$} $B[i, j-1] + A[j]$
            \EndFor
        \EndFor
        \\
        \Ret $B$
        \EndFunction
    \end{algorithmic}
\end{algorithm}

\end{document}

