\documentclass{ctexart}
\usepackage{amsmath, amsfonts, amssymb}
\usepackage{geometry}
\usepackage{IEEEtrantools}
% \special{dvipdfmx:config z 0} % delete this when release

\geometry{a4paper,scale=0.8}

\title{数据结构与算法~作业1}
\author{庄嘉毅}
\date{September 2022}

\def\QED{\hfill $\square$}
\def\st{\textrm{s.t.}\,}
\DeclareMathOperator{\e}{e}

\everymath{\displaystyle}
\linespread{2}

\begin{document}

\maketitle

\paragraph*{1} 计算下面程序中赋值语句的执行次数.
\begin{align*}
    T(n) & = \sum_{1\le i<n} \sum_{i\le j \le n}1 \\
         & =\sum_{i=1}^{n-1} n-i+1                \\
         & = \frac{(n+2)(n-1)}{2}
\end{align*}

\paragraph*{2} 已知下列运算时间函数,写出其大$O$表示的运算时间.

\subparagraph*{(1)} $T(n)=3n^3+10n^2+2n=O(n^3)$.

\subparagraph*{(2)}
\begin{align*}
    \frac{T(n)}{2^n} & = \frac{T(n-1)}{2^{n-1}} + \frac{1}{2^n}                     \\
                     & = \frac{T(n-2)}{2^{n-2}} + \frac{1}{2^{n-1}} + \frac{1}{2^n} \\
                     & = \cdots                                                     \\
                     & = \frac{T(1)}{2} + \frac{1}{2^2} + \cdots + \frac{1}{2^n}    \\
                     & = \frac{1}{2} + \frac{1}{2^2} + \cdots + \frac{1}{2^n}       \\
                     & = 1 - \frac{1}{2^n}                                          \\
\end{align*}

因此, $T(n)=2^n-1=O(2^n)$.

\paragraph*{3} 证明:

\subparagraph*{(1)} 易知 $\frac{f(n)+g(n)}{2}\le \max\{f(n),g(n)\} \le f(n)+g(n)$,
因此 $\max\{f(n),g(n)\}=\Theta (f(n)+g(n))$.

\QED

\subparagraph*{(2)} 对任意 $n\in\mathbb{N}$, 由$a>b>1$, 有 $a^n>b^n$, 因此 $b^n=O(a^n)$.

对任意 $n_0\in\mathbb{N}, c\in\mathbb{R}$,
令 $n=\min\left\{n_0,\left\lfloor \log_{\frac{a}{b}} c\right\rfloor+1\right\}$,
有 $\frac{a^n}{b^n}=\left(\frac{a}{b}\right)^n>\left(\frac{a}{b}\right)^{\log_{\frac{a}{b}} c}=c$,
因此 $a^n > c b^n$, 故 $a^n\ne O(b^n)$. \QED

\end{document}
