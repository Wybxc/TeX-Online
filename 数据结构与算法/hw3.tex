\documentclass{ctexart}
\usepackage{amsmath, amsfonts, amssymb}
\usepackage{geometry}
\usepackage{algorithm, algorithmicx, algpseudocode}
\usepackage{IEEEtrantools}
% \special{dvipdfmx:config z 0} % delete this when release

\geometry{a4paper,scale=0.8}

\title{数据结构与算法~作业3}
\author{庄嘉毅}
\date{September 2022}

\def\QED{\hfill $\square$}
\def\st{\textrm{s.t.}\,}
\DeclareMathOperator{\e}{e}
\def\nil{\textbf{nil}}
\def\And{ \textbf{and} }
\def\Not{\textbf{not} }

% \everymath{\displaystyle}
% \linespread{2}

\begin{document}

\maketitle

\paragraph*{1} ~

\begin{algorithm}[ht]
    \renewcommand{\algorithmicrequire}{\textbf{Input:}}
    \renewcommand{\algorithmicensure}{\textbf{Output:}}
    \caption{Simulate a queue using two stacks}
    \begin{algorithmic}[1]
        \Require Two stacks $S_1$ and $S_2$
        \Procedure{Enqueue}{$x$}
        \State \Call{Push}{$S_1, x$}
        \Comment{Push $x$ to the back of $S_1$}
        \EndProcedure
        \vspace*{1em}

        \Procedure{Dequeue}{$x$}
        \While{\Not \Call{Empty}{$S_1$}}
        \Comment{Move all elements from $S_1$ to $S_2$}
        \State \Call{Pop}{$S_1, a$}
        \State \Call{Push}{$S_2, a$}
        \EndWhile
        \State \Call{Pop}{$S_2, y$}
        \Comment{Now the top of $S_2$ is the first element}
        \While{\Not \Call{Empty}{$S_2$}}
        \Comment{Move all elements back to $S_1$}
        \State \Call{Pop}{$S_2, a$}
        \State \Call{Push}{$S_1, a$}
        \EndWhile
        \State \Return $y$
        \EndProcedure
        \vspace*{1em}

        \Procedure{QueueEmpty}{}
        \State \Return \Call{Empty}{$S_1$}
        \Comment{Note that $S_2$ is always empty between operations}
        \EndProcedure
    \end{algorithmic}
\end{algorithm}

\paragraph*{2} 证明: 考虑一个出入栈操作序列 $S$,
其中 $S_i \in \{\mathrm{Push}, \mathrm{Pop}\}, i=1,2,\ldots,2n$.
$S$ 是合法的当且仅当
\begin{gather*}
    \sum_{i=1}^k [S_i=\mathrm{Push}] \ge \sum_{i=1}^k [S_i=\mathrm{Pop}],\\
    \textrm{where } k=1,2,\ldots,2n,\\
    [A] = \begin{cases}
        1 & A \text{ is true},  \\
        0 & A \text{ is false},
    \end{cases}
\end{gather*}
且
\begin{gather*}
    \sum_{i=1}^{2n} [S_i=\mathrm{Push}] = \sum_{i=1}^{2n} [S_i=\mathrm{Pop}].
\end{gather*}
下面说明, 合法的 $S$ 与合法的出栈顺序一一对应.

显然, 从出入栈操作序列 $S$ 到出栈顺序 $A$ 存在天然的满射.
只需说明此映射是单射.

若两个不同的出入栈操作序列 $S^1, S^2$ 对应同一个出栈顺序 $A$, 
考虑最小的使得 $S^1_i\ne S^2_i$ 的 $i$. 
不妨设 $S^1_i=\mathrm{Push}, S^2_i=\mathrm{Pop}$,
并记由 $S^1_i$ 对应的入栈元素为 $a$, 由 $S^2_i$ 对应的出栈元素为 $b$.
那么在时刻 $i$ 后, $S^1$ 对应的状态中 $a$ 在 $b$ 之前出栈, 
而 $S^2$ 对应的状态中 $a$ 在 $b$ 之后出栈.
因此, $S^1$ 和 $S^2$ 不可能对应同一个出栈顺序 $A$.

因此, 合法的出入栈操作序列 $S$ 的个数即为卡塔兰数
\begin{gather*}
    C_n = \frac{1}{n+1} \binom{2n}{n}.
\end{gather*}

\paragraph*{3} 

\subparagraph*{必要性} 假设存在 $i<j<k$ 使得 $P_j<P_k<P_i$, 
那么在 $P_i$ 出栈时, 由于 $P_i$ 更大, 所以 $P_j, P_k$ 已经完成入栈,
并且由于 $P_j, P_k$ 在 $P_i$ 之后出栈, 所以此时 $P_j, P_k$ 仍在栈内.
根据先入后出, $P_j<P_k$, 因此 $P_j$ 先入栈, 其出栈应在 $P_k$ 之后,
与 $j<k$ 矛盾. 因此合法的出栈序列中不存在这样的 $i, j, k$.

\subparagraph*{充分性} 使用数学归纳法. 当 $n=1,2,3$ 时, 易知结论成立.
当 $n\ge 4$ 时, 假设结论对 $1,2,\ldots,n-1$ 均成立,
讨论序列 $P$ 中 1 的位置, 记 $P_j=1$.

若 $j=1$, 则 $P_2, P_3, \ldots, P_n$ 满足条件, 根据归纳假设, 
为合法的出栈序列, 进而 $P$ 为合法的出栈序列.

同理, 若 $j=n$, 则 $P_1, P_2, \ldots, P_{n-1}$ 满足条件,
同上根据归纳假设, 可知 $P$ 为合法的出栈序列.

若 $1<j<n$, 任取 $1\le i < j < k \le n$, 有 $P_k > P_i$, 
记 $P_1, P_2, \ldots, P_{j-1}$ 中最大数为 $m$, 那么根据归纳假设,
$P_1, P_2, \ldots, P_{j-1}$ 为 $2,\ldots, m$ 的一个合法的出栈序列,
$P_{j+1}, P_{j+2}, \ldots, P_n$ 为 $m+1, m+2, \ldots, n$ 的一个合法的出栈序列,
因此 $P$ 为合法的出栈序列.

\QED

\end{document}